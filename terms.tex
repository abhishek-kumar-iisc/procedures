\chapter{Terms and Conventions}

%%%%%%%%%%%%%%%%%%%%%%%%%%%%%%%%%%%%%%%%%%%%%%%%%%%%%%%%%%%%%%%%%%%%%%%%%%%

{\color{red}{

\section{Document Definitions}

This document defines rules and conventions for the creation of:

\begin{itemize}
\item {\bf MPI Standard Document:} the main document defining the
  Message Passing Interface standard.
\item {\bf MPI Companion Documents:} additional documents published by
  the MPI Forum in addition to the MPI Standard Document. These
  documents are ratified by the MPI Forum, but are not part of the
  official MPI Standard Document.
\end{itemize}

Both types of documents are marked with monotonically increasing version numbers using a major/minor version number scheme.

\section{Definitions of Roles}

The following defines the roles of the people or groups of people
involved in the MPI standardization process:

\begin{itemize}
\item {\bf MPI Forum:} the group of people actively involved in the
  standardization process by participation in physical meetings,
  participation in MPI working groups, and / or chapter committees.
\item {\bf MPI Forum Chair:} is responsible for organizing the agenda
  for physical MPI Forum meetings as well as the activities leading to the
  publication of MPI Standard and MPI Companion Documents. The Forum
  Chair also maintains the overall outside presence of the MPI Forum.
\item {\bf MPI Forum Secretary:} is responsible for organizing and
  recording ballots as well as artifacts from the physical MPI Forum
  meetings.
\item {\bf MPI Standard Document Editor:} is responsible for both
  maintaining the overall document and its repository, and for
  publishing newly ratified versions of the MPI documents.
\item {\bf Chapter Committee Chair (sometimes referred to as ``Chapter
  Author''):} is responsible for implementing and organizing reviews
  for approved changes into their respective chapter(s).
\item {\bf Chapter Committee:} assists the Chapter Committee Chair in
  implementing and reviewing changes for the respective chapters.
\item {\bf Working Group:} group of people working on individual,
  possibly cross-cutting topics that can lead to proposed changes for
  the MPI Standard Document. Working groups can be established at MPI
  Forum meetings once support from at least four IMOVE organizations.
\item {\bf Working Group Chair:} is responsible for organizing the
  work in the Working Group, reporting to the MPI Forum on progress in
  the working group, maintains the outside presence of the Working
  Group, and organizing regular Working Group meetings.
\end{itemize}

\section{Ballot Definitions}

\begin{itemize}
\item {\bf Physical MPI Forum Meeting:} An open meeting of the entire
  MPI Forum in a physical location (vs.\ a teleconference or other
  virtual meeting).  In-person attendance to the meeting is open to
  all organizations in the MPI Forum as well as the general public.

\item {\bf Organization:} A business entity that sends one or more
  representatives to a physical MPI Forum meeting.

\item {\bf Registration:} Individuals register for each physical MPI
  Forum meeting that they will attend.  At the time of registration,
  individuals declare which organization they will represent at that
  meeting.

\item {\bf Overall Organization Eligibility (OOE):} An organization is
  generally eligible to vote if it has registered and had one or more
  representatives physically present at two out of the last three
  physical MPI Forum meetings (including the current meeting).

\item {\bf Individual Meeting Organization Voting Eligibility
    (IMOVE):} An organization is eligible to vote at a specific
  physical MPI Forum meeting if all of the following are true:
  \begin{itemize}
  \item The organization is OOE.
  \item An individual representing this organization registered for
    that specific physical MPI Forum meeting before the first ballot
    occured.
  \item The organization had at least one of its representatives
    physically present during that specific physical MPI
    Forum meeting.
  \end{itemize}
  Once an organization becomes IMOVE for a specific physical MPI Forum
  meeting, that organization stays IMOVE for the remainder of that
  specific physical MPI Forum meeting.  For example, if an
  organization's only representative leaves the meeting, that
  organization still remains IMOVE.

\item {\bf Meeting Quorum:} Quorum is established at a physical MPI
  Forum meeting when more than $\nicefrac{2}{3}$ of OOE organizations
  have registered for that meeting.

\item {\bf Individual Ballot Quorum:} Quorum is established for an
  individual ballot when more than $\nicefrac{3}{4}$ of IMOVE
  organizations at the meeting cast a vote (vs.\ abstain).  The number
  of IMOVE organizations is counted at the beginning of each ballot.
\end{itemize}


}} % color red
