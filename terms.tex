\chapter{Terms and Conventions}

%%%%%%%%%%%%%%%%%%%%%%%%%%%%%%%%%%%%%%%%%%%%%%%%%%%%%%%%%%%%%%%%%%%%%%%%%%%

\section{Document Definitions}

This document defines rules and conventions for the creation of:

\begin{itemize}
\item {\bf OpenSHMEM Specification Document:} the main document defining the
  OpenSHMEM Application Programming Interface specification.
\end{itemize}

OpenSHMEM specification documents are marked with monotonically increasing
version numbers using a major/minor version number scheme.

\section{Definitions of Roles}

The following defines the roles of the people or groups of people
involved in the OpenSHMEM specification process:

\begin{itemize}
\item {\bf OpenSHMEM Committee:} The group of
  \oldtext{people}
  \newtext{contributors}
  actively involved in the
  specification process by participation in official meetings,
  participation in OpenSHMEM working groups, and \replacetext{chapter}{section} committees.
\item {\bf OpenSHMEM Committee Chair:} Is responsible for organizing the agenda
  for official OpenSHMEM Committee meetings as well as the activities leading to the
  publication of the OpenSHMEM Specification. The Committee
  Chair also maintains the overall outside presence of the OpenSHMEM Committee.
\item {\bf OpenSHMEM Committee Secretary:} Is responsible for organizing and
  recording ballots as well as artifacts from the official OpenSHMEM Committee
  meetings.
\item {\bf OpenSHMEM Specification Document Editor:} Is responsible for both
  maintaining the overall document and its repository, and for
  publishing newly ratified versions of the OpenSHMEM documents.
\item {\bf \replacetext{Chapter}{Section} Committee Chair (sometimes referred to as ``\replacetext{Chapter}{Section}
  Author''):} Is responsible for implementing and organizing reviews
  for approved changes into their respective \replacetext{chapter}{section}(s).
\item {\bf \replacetext{Chapter}{Section} Committee:} Assists the \replacetext{Chapter}{Section} Committee Chair in
  implementing and reviewing changes for the respective \replacetext{chapter}{section}s.
\item {\bf Working Group:} Group of people working on individual,
  possibly cross-cutting topics that can lead to proposed changes for
  the OpenSHMEM Specification Document.
  \oldtext{Working groups can be established at OpenSHMEM
  meetings once at least four organizations indicate
  support for that proposed Working Group.}
  \newtext{Working groups are established and dissolved through an official
  ballot at an OpenSHMEM meeting.  In order to enable rapid exploration of new
  topics, ballots to establish new working groups can be added to the agenda at
  any time before the first vote at a given meeting.  However, committee
  members are encouraged to schedule such ballots at least two weeks prior to
  the start date of the official OpenSHMEM meeting.}
\item {\bf Working Group Chair:} Is responsible for organizing the
  work in the Working Group, reporting to the OpenSHMEM Committee on progress in
  the working group, maintaining the outside presence of the Working
  Group, and organizing regular Working Group meetings.
\end{itemize}

\section{Ballot Definitions}

\begin{itemize}
\item {\bf Official OpenSHMEM Committee Meeting:} An open meeting of the entire
  OpenSHMEM Committee.  Attendance to the meeting is open to
  all organizations in the OpenSHMEM Committee as well as the general public.

\item {\bf Organization:} A business entity that sends one or more
  representatives to a official OpenSHMEM Committee meeting.

\item {\bf Overall Organization Eligibility (OOE):} An organization is
  generally eligible to vote if it
  \newtext{has signed the OpenSHMEM contributor agreement and}
  is considered to be a member in good
  standing by Open Source Software Solutions, Inc. (OSSS).  A list of eligible
  organizations will be made publicly available (e.g. on the OpenSHMEM website)
  before the ballot deadline for any official meeting.

\item {\bf Meeting Quorum:} Quorum is established at a official OpenSHMEM
  Committee meeting when more than $\nicefrac{2}{3}$ of OOE organizations
  are present for that meeting.

\item {\bf Individual Ballot Quorum:} Quorum is established for an
  individual ballot when more than $\nicefrac{3}{4}$ of OOE
  organizations at the meeting cast a vote (vs.\ abstain).  The number
  of OOE organizations is counted at the beginning of each ballot.
  \newtext{Alternatively, the number of OOE organizations may be verified after
  each ballot to determine if quorum was met.}
\end{itemize}
