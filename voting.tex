\chapter{Voting Rules}

%%%%%%%%%%%%%%%%%%%%%%%%%%%%%%%%%%%%%%%%%%%%%%%%%%%%%%%%%%%%%%%%%%%%%%%%%%%

\section{Intent}

This chapter was written with the following goals in mind:

\begin{enumerate}
\item Provide clear, unambiguous definitions and procedures for voting
  on general text proposals, the final
  OpenSHMEM Specification Document, and changes to this document.
\item Enforce a high degree of consensus before text is accepted into
  the OpenSHMEM Specification Document.
\item Specify a process that ensures a high quality
  OpenSHMEM Specification Document and that allows for fixes to the
  OpenSHMEM Specification Document for issues found in final review stages.
\item Disallow arbitrary abuse of voting procedures.
\end{enumerate}

This document only details {\em official ballot} voting definitions
and procedures.  Unofficial voting procedures, such as ``straw''
votes, are outside the scope of this document.

%%%%%%%%%%%%%%%%%%%%%%%%%%%%%%%%%%%%%%%%%%%%%%%%%%%%%%%%%%%%%%%%%%%%%%%%%%%

\section{Procedures}

%-------------------------------------------------------------------------

\subsection{Official Ballot Voting}
\label{subsec:official-ballot-voting}

Official ballot voting and formal readings occur only at official OpenSHMEM
Committee meetings where a meeting quorum has been established.

All official ballots must be announced and scheduled at least two
weeks prior (four weeks prior in case of votes for a final OpenSHMEM Specification
Document) to the start date of the official OpenSHMEM Committee meeting at
which they will be held.  The dates/times for official ballots will
not change after two weeks prior to the beginning of the meeting to
allow attendees to schedule their attendance appropriately.
As an exception, a meeting can be canceled with unanimous consent of
all OOE organizations. Any business scheduled for the canceled meeting can be
rescheduled for a future meeting, provided that it meets the scheduling
constraints for that meeting at the time it is rescheduled (e.g. two or four
weeks prior to the start of the meeting).

For each official ballot, each OOE organization is individually
polled for their vote.  The designated representative of an OOE
organization may vote ``yes,'' vote ``no,'' or abstain from voting.
Proxies are not permitted.  If no representative of an OOE
organization is present at the time of the ballot, that
organization has implicitly abstained.

A ballot passes if:

\begin{enumerate}
\item The ballot meets the requirements for the individual ballot
  quorum, and
\item The number of ``yes'' votes is more than $\nicefrac{3}{4}$ of
  the sum of ``yes'' and ``no'' votes.
\end{enumerate}

\begin{rationale}
  The first condition prevents large numbers of abstentions from
  skewing results.  The second condition sets a high requirement for
  consensus before a ballot will pass.
\end{rationale}

Note that if a ballot fails to meet the required individual ballot
quorum, the ballot can be re-cast one time at the same official OpenSHMEM
Committee meeting.  The ballot may also be deferred to a subsequent
official OpenSHMEM Committee meeting.  Specifically: failing to establish the
individual ballot quorum does not mean that the ballot failed.

%-------------------------------------------------------------------------

\subsection{Proposals}
\label{subsec:general-text-proposals}

Proposals use the following process to be accepted into
an OpenSHMEM Specification Document:

\begin{enumerate}
\item Have a formal reading at a official OpenSHMEM Committee meeting where the
  meeting quorum has been met.
  \begin{enumerate}
  \item The final text of the proposal to be read must be made
    publicly available via the general OpenSHMEM Committee broadcast email list
    at least two weeks prior to the start date of the official OpenSHMEM
    Committee meeting at which it is to be formally read.
  \item The formal reading must be scheduled on the official OpenSHMEM Committee
    meeting's agenda at least two weeks prior to the meeting's start
    date.
  \item There is no criteria for ``passing'' or ``failing'' a formal
    reading.  It is up to the proposal's author(s) to decide whether
    to bring the proposal up for a formal ballot at a subsequent
    meeting.
  \end{enumerate}

\item Pass an official ballot at a official OpenSHMEM Committee meeting.
  \begin{enumerate}
  \item A proposal's ballot can only be conducted after its
    formal reading.
  \item A proposal's ballot must be conducted at a different
    official OpenSHMEM Committee meeting than which it was formally read.
  \end{enumerate}

\item Changes to proposal text after it was made available for the
  formal reading (i.e., at least two weeks prior to the start date of
  the official OpenSHMEM Committee meeting at which it was read) are permitted
  in some cases:
  \begin{enumerate}
  \item After the formal reading and before the ballot, changes are permitted if the text
    delta is presented at a official OpenSHMEM Committee meeting and approved
    via a special formal ballot of OOE organizations at that
    meeting:
    \begin{enumerate}
    \item The ballot meets the requirements for the individual
      ballot quorum, and
    \item There are zero ``no'' votes.
    \end{enumerate}

    \begin{rationale}
      The first condition prevents a large number of abstentions.
      The second condition ensure that all non-abstaining
      organizations are unanimous in their consent of the text
      changes.
    \end{rationale}

    If the special ballot fails, the original text of the proposal
    is used.

  \item After the ballot, text changes that do not change the
    semantics of the proposal are permitted with the unanimous consent
    of the relevant section committee(s).
  \end{enumerate}
\end{enumerate}

Proposals may be voluntarily withdrawn at any time before the
ballot passes.

Ballots may be deferred to a subsequent official OpenSHMEM Committee meeting in
the following cases:

\begin{enumerate}
\item Before the ballot is conducted, the proposal author requests a
  deferral to the next official OpenSHMEM Committee meeting.
\item When the ballot is conducted, it fails to meet the individual
  ballot quorum.
\end{enumerate}

If a proposal fails either its ballot, or if a proposal is
withdrawn, it must perform the entire procedure again (i.e., start
over with a formal reading).  If a ballot fails to establish its
per-ballot quorum, it may be re-cast within the timeframes specified
above.

%-------------------------------------------------------------------------

\subsection{Process to Ratify an OpenSHMEM Specification Document}

Once a series of changes are voted in by
the OpenSHMEM Committee using the processes above, the Committee can publish a new
revision of the OpenSHMEM Specification Document.  This could be a new minor or
major version of the standard; the process below applies to either.
The Committee Chair, after consulting with the members of the Committee,
initiates this process.
In addition, the committee chair divides the specification document into
sections and forms section committees.

The ratification process of any OpenSHMEM Committee Document starts after the
end of the last official OpenSHMEM Committee meeting where changes were voted
into that Document, and typically spans two subsequent official OpenSHMEM
Committee meetings:

\begin{itemize}
\item Release Candidate Meeting (RCM)
\item Final Ratification Meeting (FRM)
\end{itemize}

Ratification procedures are as follows:

\begin{enumerate}
\item Prior to four weeks before the start of the RCM:
  \begin{itemize}
  \item Section Committee Chairs integrate approved changes and/or
    minor, non-semantic fixes to their sections into the OpenSHMEM Specification
    Document.
  \item Section Committees review changes to their sections to ensure
    that approved changes have been integrated accurately into the OpenSHMEM
    Specification Document.
  \item Section Committees may also find problems with approved
    changes that require further deliberation by the Committee.  Such
    problems must be itemized for review by the Committee.
  \end{itemize}

\item At least four weeks before the start of the RCM:
  \begin{enumerate}
  \item Section Committee Chairs determine whether there have been
    any changes to their sections since the last published version.
  \item If there have been changes since the last published version,
    Section Committee Chairs publish the following for the Committee
    members to review:
    \begin{itemize}
      \item Release Candidate Drafts (in PDF form) of their sections.
      \item Changes to the section since the last published version
        (preferably in the form of a colorized diff, or a marked up
        PDF, or some other easily-reviewable format showing the
        changes).
      \item List of still-unresolved problems, including (but not
        limited to) problems with or mistakes in approved changes.
    \end{itemize}
  \item If there have been no changes since the last
    published version, Section Committee Chairs inform the OpenSHMEM Committee
    Chair and the OpenSHMEM Specification Document Editor of this fact.
  \end{enumerate}

\item After all Section Committee Chairs have published their section
  drafts, but no later than three weeks before the start of the RCM:
  \begin{enumerate}
  \item The OpenSHMEM Specification Document Editor publishes a Release Candidate
    Draft of the entire OpenSHMEM Specification Document (in PDF form), including
    all the changes from all Section Committees.
  \end{enumerate}

\item In the four-week window before the start of the RCM:
  \begin{enumerate}
  \item OpenSHMEM Committee members review all the material published by the
    Section Committee Chairs and OpenSHMEM Specification Document Editor.
  \item Section Committees continue to work on still-unresolved
    issues.  {\em Any} changes to text after the Section
      Committee Chairs publish their section drafts at the four-week
    window must be specifically discussed with the Committee at the RCM.
  \end{enumerate}

\item At the RCM:
  \begin{enumerate}
  \item All Section Committee Chairs (or their designees) read their
    sections for the entire Committee.  The focus of the readings is the
    changes that have occurred since the last released version and the last read version of a ratified proposal (as
    opposed to verbally reading the entire section word-for-word).
  \item Items that must be specifically itemized and discussed with
    the Committee during these readings include:
    \begin{itemize}
    \item Any unresolved issues found in implementing approved
      changes.
    \item Any technical issues found with approved changes or with the
      existing OpenSHMEM Specification Document.
    \item Any changes that were made within four weeks of the
      beginning of the RCM.
    \end{itemize}

  \item The OpenSHMEM Committee collectively reviews the entire Release
    Candidate Draft OpenSHMEM Specification Document, looking for problems such
    as (but not limited to):
    \begin{itemize}
    \item Formatting and whitespace problems, spelling errors, and
      other typos.  Such problems should be itemized and can be fixed
      at the meeting by Section Committees and/or the OpenSHMEM Specification
      Document Editor.
    \item Logical inconsistencies in the overall document, or problems
      with approved changes.
    \end{itemize}

  \item The OpenSHMEM Committee Chair compiles a list of all still-unresolved
    issues that will be fixed before this release of the OpenSHMEM Specification
    Document.
    \begin{itemize}
    \item Committee members are encouraged to only allow
      ``errata''-quality items on the list of still-unresolved
      issues.  Larger items should either delay the ratification
      process or be deferred to a future version of the OpenSHMEM Specification
      Document.
    \end{itemize}

  \item\label{voting:rcm:ballot} Per section~\ref{subsec:general-text-proposals}, a 
    ballot is conducted on ratifying the entire Release Candidate
    Draft OpenSHMEM Specification Document {\em along with} the listing of all
    still-unresolved issues and whitespace/spelling/typo fixes created
    in the previous steps.
    \begin{itemize}
    \item If the ballot fails, the entire procedure must be repeated,
      possibly starting a new RCM at the next official meeting.
    \end{itemize}

  \item The ratification can be ``fast tracked'' if the following
    conditions are true:

    \begin{itemize}
    \item The ballot from (\ref{voting:rcm:ballot}) at the
      same meeting passed.
    \item  The list of still-unresolved issues is empty.
    \item The Committee resolved all other minor issues, such as
      formatting and whitespace problems, spelling errors, and other
      typos, and the OpenSHMEM Specification Document Editor has produced a new
      Release Candidate Document containing all these fixes.
    \item After a new Release Candidate Document is available, the
      Committee decides, via special formal ballot, to ``fast track'' the
      ratification.  The ballot passes if:
      \begin{enumerate}
      \item The ballot meets the requirements for the individual
        ballot quorum, and
      \item There are zero ``no'' votes.
      \end{enumerate}
    \end{itemize}

    \item If all conditions are met, the ratification is fast tracked,
      steps \ref{voting:slow-track-begin}) through
        (\ref{voting:slow-track-end}) are skipped, and step
        (\ref{voting:fast-track-begin}) can be performed at the RCM.
  \end{enumerate}

\item\label{voting:slow-track-begin} Prior to four weeks before the
  start of the FRM:
  \label{subsec:official-ballot-voting:t-4weeks-frm}
  \begin{itemize}
  \item Section Committees and Working Groups work on resolving the
    issues in the list of open issues, integrate changes
    into the Release Candidate Document, and review any
    changes made.
  \end{itemize}

\item At least four weeks before the start of the FRM:
  \begin{itemize}
  \item Section Committee Chairs with changes to their sections since
    the RCM publish the final draft of their sections.
  \item Section Committee Chairs publish the list of all
    changes made since the RCM, including changes made based on the
    list of open issues.
  \end{itemize}

\item After all Section Committee Chairs with changes to their
  sections have published updated section drafts, but no later than
  three weeks before the start of the FRM:
  \begin{enumerate}
  \item The OpenSHMEM Specification Document Editor freezes the Release Candidate
    Document and publishes it to the OpenSHMEM Committee.
  \end{enumerate}

\item At the FRM:
  \begin{enumerate}
  \item The OpenSHMEM Committee Secretary conducts a ballot for each individual
    change that originated from the list of open issues decided upon
    at the RCM (and was completed before the four-week window).
    Ballots that fail must have their changes reverted.
  \item\label{voting:slow-track-end} The OpenSHMEM Committee Secretary conducts
    a series of ballots for all other changes made since the RCM.  In
    addition to the procedures listed in
    Section~\ref{subsec:official-ballot-voting}, if any ``no'' votes
    are recorded in the ballot for a given change, this change must be
    reverted.
  \item\label{voting:fast-track-begin} The OpenSHMEM Committee Chair compiles a
    list of all still-unresolved issues that could be fixed before
    this release of the OpenSHMEM Specification Document.

    \item If any issues remain on the list of still-unresolved issues,
      the OpenSHMEM Committee Secretary conducts a ballot to decide whether
      these issues delay ratification.
      \begin{itemize}
      \item If the ballot passes, the next official OpenSHMEM Committee meeting
        will repeat this FRM; this process jumps back to
        step~\ref{subsec:official-ballot-voting:t-4weeks-frm}.
      \item If the ballot fails, the Release Candidate Document
        remains unchanged.
      \end{itemize}
    \item The OpenSHMEM Committee Secretary conducts a final ballot on the
      entire Document.
      \begin{itemize}
      \item If the ballot passes, the OpenSHMEM Specification Document Editor adds
        a date stamp to the Document.
        As part of the licensing agreement with Hewlett Packard
        Enterprise~(HPE), the Document is then submitted to HPE and OSSS for
        approval.  Upon receiving approval, the OpenSHMEM Document Editor
        publishes it to the OpenSHMEM web site.
      \item If the ballot fails, the entire ratification process must be
        repeated.
      \end{itemize}
  \end{enumerate}
\end{enumerate}

%-------------------------------------------------------------------------

\subsection{Changing These Rules}

The procedure for changing these rules is essentially the same as for
other proposals: publish the proposed change at least two weeks prior
to a official OpenSHMEM Committee meeting and then pass one official ballot.
The new rules take effect as soon as they are approved/voted in by the
OpenSHMEM Committee.

% LocalWords:  OOE
