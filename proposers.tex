\chapter{Suggestions for Proposers}

The following are several suggestions to consider before raising a
proposal to the OpenSHMEM Committee:

\begin{enumerate}
\item Socialize your proposal among all the relevant \replacetext{chapter}{section}
  committees, working groups, other relevant committee members, and real-world users.  Get
  feedback and buy-in from as many people as possible.

\item Ensure that your proposal:
  \begin{enumerate}
  \item Is not ``syntactic sugar'' for something that could be
    implemented outside of an OpenSHMEM implementation.
  \item Represents a ``best practice.''
  \item Is useful on a wide variety of platforms / architectures, both
    today and in the conceivable future.
  \item Is not an ephemeral use case.
  \end{enumerate}

\item Be prepared to cite concrete use cases and/or applications that
  can use the functionality in your proposal.

\item Implementations of proposals are strongly encouraged, especially
  for ``non-trivial'' proposals.  The most highly valued
  implementations are ones that:
  \begin{enumerate}
  \item Show a performance or functionality benefit that cannot be
    accomplished outside of an OpenSHMEM implementation.
  \item Can be implemented on a wide variety of platforms /
    architectures.
  \end{enumerate}

\item Proposal quality issues:
  \begin{enumerate}
  \item Use a similar writing style to the rest of the OpenSHMEM
    specification document.
  \item Get the proposal proofread by a native English speaker.
  \item Ensure that the proposal fits in well with the overall OpenSHMEM
    specification document.
  \end{enumerate}

\item Don't let too much time elapse between the formal reading and
  ballots.

\end{enumerate}
